%%%%%%%%%%%%%%%%%%%%%%%%%%%%%%%%%%%%%%%%%%%%%%%%%%%%%%%%%%%%%%%%%%%%%%%%%%%%%%%%
%                                                                              % 
% Autor:  Dominik Bacher                                                       %
% Datum:  26.01.2009                                                           % 
% Datei:  praesentation.tex                                                    %
%                                                                              %
% Letzte Änderung: Tobias Dreher                                               %
% Datum:           06.03.2011                                                  %
%                                                                              % 
% Quelle: http://www2.informatik.hu-berlin.de/~mischulz/beamer.html            % 
%                                                                              %
%%%%%%%%%%%%%%%%%%%%%%%%%%%%%%%%%%%%%%%%%%%%%%%%%%%%%%%%%%%%%%%%%%%%%%%%%%%%%%%%

\documentclass{beamer}
%\documentclass[handout,xcolor=table]{beamer}
%\documentclass[class=article,a4paper,11pt,twoside]{beamer}% handout

%%%%%%%%%%%%%%%%%%%%%%%%%%%%%%%%%%%%%%%%%%%%%%%%%%%%%%%%%%%%%%%%%%%%%%%%%%%%%%%%
% Sprache, Schriftsatz und Schriftart
\usepackage[utf8]{inputenc}
\usepackage[T1]{fontenc}
\usepackage{setspace}
\usepackage{lmodern} %Schriftart

%%%%%%%%%%%%%%%%%%%%%%%%%%%%%%%%%%%%%%%%%%%%%%%%%%%%%%%%%%%%%%%%%%%%%%%%%%%%%%%%
%Farbdefinitionen
\usepackage{xcolor}
\usepackage{color}
\definecolor{lightblue}{rgb}{0.8,0.85,1}
\definecolor{lightred}{rgb}{1.0,0.5,0.5}
\definecolor{lightgray}{rgb}{0.8,0.8,0.8}
\definecolor{darkblue}{HTML}{004494}
\definecolor{darkred}{rgb}{.6,0,0}
\definecolor{darkgreen}{rgb}{0,.6,0}
\definecolor{red}{rgb}{.98,0,0}

%%%%%%%%%%%%%%%%%%%%%%%%%%%%%%%%%%%%%%%%%%%%%%%%%%%%%%%%%%%%%%%%%%%%%%%%%%%%%%%%
% Tabellen
\usepackage{longtable}  % fuer Tabellen die ueber die Seitengrenze gehen
\usepackage{tabularx}
\renewcommand{\arraystretch}{1.4} % Zeilenabstand

%%%%%%%%%%%%%%%%%%%%%%%%%%%%%%%%%%%%%%%%%%%%%%%%%%%%%%%%%%%%%%%%%%%%%%%%%%%%%%%%
% Absatz
\parskip 6pt plus 1pt minus 1pt	% Setzt den vertikalen Abstand zwischen
					% Absaetzen auf 12 pt. Das Plus und
					% Minus fuegt Glue ein, d. h. TeX darf
					% den Abstand um einen Punkt
					% vergroessern oder verkleinern, um ein
					% gutes Layout zu erzeugen

\parindent 0pt      % Setzt die Einrückung der ersten Zeile auf 0 Pt

\usepackage{url}
\usepackage{lastpage}

%%%%%%%%%%%%%%%%%%%%%%%%%%%%%%%%%%%%%%%%%%%%%%%%%%%%%%%%%%%%%%%%%%%%%%%%%%%%%%%%
% Mehrere Spalten        \begin{multicols}{2}[] \end{multicols}
\usepackage{multicol}
\setlength{\columnsep}{2em}

%%%%%%%%%%%%%%%%%%%%%%%%%%%%%%%%%%%%%%%%%%%%%%%%%%%%%%%%%%%%%%%%%%%%%%%%%%%%%%%%
% Grafik
\usepackage{epsfig}            
\usepackage{graphicx}
\usepackage{pgf}
%\usepackage{subfigure}
%\usepackage{subfig}
\usepackage{capt-of}
\usepackage{wrapfig}
\usepackage{sidecap}
%\usepackage{pstricks}

%%%%%%%%%%%%%%%%%%%%%%%%%%%%%%%%%%%%%%%%%%%%%%%%%%%%%%%%%%%%%%%%%%%%%%%%%%%%%%%%
% Farbige Hyperlinks
\usepackage{hyperref}
\hypersetup{linktocpage=true,
        colorlinks=true,
        linkcolor=darkblue,
        citecolor=darkblue}

%%%%%%%%%%%%%%%%%%%%%%%%%%%%%%%%%%%%%%%%%%%%%%%%%%%%%%%%%%%%%%%%%%%%%%%%%%%%%%%%
% Mathe
\usepackage{amsmath}
\usepackage{amssymb}

\usepackage{booktabs}
\usepackage{dcolumn}

%%%%%%%%%%%%%%%%%%%%%%%%%%%%%%%%%%%%%%%%%%%%%%%%%%%%%%%%%%%%%%%%%%%%%%%%%%%%%%%%
% Sonderzeichen
\usepackage{pifont}

%%%%%%%%%%%%%%%%%%%%%%%%%%%%%%%%%%%%%%%%%%%%%%%%%%%%%%%%%%%%%%%%%%%%%%%%%%%%%%%%
% Listings
\usepackage{listings}

% Programmiersprache
\lstset{language=PHP}

% Farben
\lstset{keywordstyle=\color{blue},
        basicstyle=\small\ttfamily,
        stringstyle=\color{red}\ttfamily}

% Rahmen
%\lstset{frame=[t][l][r][b]}

% Zeilennummern
\lstset{numbers=left,
        numberstyle=\tiny,
        stepnumber=1,
        numbersep=10pt}

% Einruecken
\lstset{breakautoindent = true,
        breakindent = 2em,
        breaklines = true,
        showspaces=false,
        showtabs=false,
        showstringspaces=false}

\newcommand{\codefile}{
        \lstset{backgroundcolor=\color{lightblue},
                numbers=left}}

\newcommand{\logfile}{
        \lstset{backgroundcolor=\color{lightgray},
                numbers=none}}

\newcommand{\xmlfile}{
        \lstset{backgroundcolor=\color{lightgray},
                numbers=none}}

% Keywords
\lstset{morekeywords={new,mysqli,prepare,bind_param,execute}}

%------------------------------------------------------------------------------
% Style der Praessentation
\mode<presentation>{
	\usetheme{Frankfurt}
			% AnnArbor | Antibes | Bergen |
			% Berkeley | Berlin | Boadilla |
			% boxes | CambridgeUS | Copenhagen |
			% Darmstadt | default | Dresden |
			% Frankfurt | Goettingen |Hannover |
			% Ilmenau | JuanLesPins | Luebeck |
			% Madrid | Malmoe | Marburg |
			% Montpellier | PaloAlto | Pittsburgh |
			% Rochester | Singapore | Szeged |
			% Warsaw

	\usecolortheme{default}
			% albatross | beaver | beetle |
			% crane | default | dolphin |
			% dove | fly | lily | orchid |
			% rose |seagull | seahorse |
			% sidebartab | structure |
			% whale | wolverine

	\usefonttheme{default}
			% default | professionalfonts | serif |
			% structurebold | structureitalicserif |
			% structuresmallcapsserif

	\useinnertheme{default}
			% circles | default | inmargin |
			% rectangles | rounded

	\useoutertheme{default}
			% default | infolines | miniframes |
			% shadow | sidebar | smoothbars |
			% smoothtree | split | tree

	% Halbtransparente Overlays:
	\setbeamercovered{transparent}

	% Abschalten der kleinen Navigationsleiste am unteren Rand:
	\beamertemplatenavigationsymbolsempty

	% Seitenzahlen in Fußzeile einfügen
	%\setbeamertemplate{footline}[frame number]
	%\setbeamertemplate{footline}{
	%	\raisebox{0.15cm}{
	%		\hspace*{0.5em} 
	%		\insertauthor, \datum
	%		\hspace*{0.805\paperwidth} %Muss evtl. angepasst werden
	%		\insertframenumber /\inserttotalframenumber
	%		}
	%}
}
\mode<article>{}
