\chapter{Einleitung}
Textunabhängige Sprechererkennung stellt schon seit Jahrzehnten ein interessanten Forschungsgebiet in Bereichen diversen Bereichen der Informatik und außerhalb dar. \cite{bib:speakerRecognition} In diesem Kapitel wird die Motivation zur Sprechererkennung, sowie die Ziele dieses Projekts und den Aufbau dieses Dokuments gezeigt.

\section{Motivation}
Einsatzgebiete gibt es beispielsweise in der Forensik. Sind Sprachaufzeichnungen vorhanden können diese mit einer Datenbank von Sprechern verglichen werden um so Personen zu identifizieren.

Ebenso können Zutrittsbeschränkungen für Sicherheitsbereiche mittels Sprechererkennung durchgeführt werden. Allerdings sind hier meist Kopplungen mit anderen Methoden notwendig, da das System leicht getäuscht werden kann, beispielsweise per Tonaufzeichnung.

Zusätzlich kann überall wo die natürliche Weise der Erkennung wichtig ist die Methoden der Sprechererkennung eingesetzt werden, beispielsweise in der Heimautomation, Automobilen und Mobilgeräten.

\section{Ziele}
Die Ziele dieser Arbeit gruppieren sich in zwei Teile. Zum einen ist ein Ziel die Implementierung einer Sprechererkennung, die Offline gute Ergebnisse erzielt. Hierbei steht der Vergleich verschiedener Algorithmen im Vordergrund. Zum anderen soll ein Online-Modus implementiert werden, der in Echtzeit den aktuellen Sprecher ermittelt.

\section{Aufbau des Dokuments}
In den folgenden Kapiteln wird näher auf die Theorie und die konkrete Implementierung des Projekts eingegangen. In Kapitel \ref{cha:ergebnisse} werden die Ergebnisse dargestellt und in Kapitel \ref{cha:diskussion} diskutiert.
