\chapter{Theorie}
Übersicht über dieses Kapitel

\section{Übersicht}
4 Schritte: Preprocessing, Training, Prediction, Analysis

\section{Preprocessing}
Übersicht über Preprocessing
\subsection{Konvertierung}
Konvertierung des Audiosignals (Mono, 16 kHz, 16 Bit, nicht immer notwendig)
\subsection{Feature Extration}
Linear Predictive Coding (LPC)\\
Mel Frequency Cepstral Coefficients (MFCC)
\subsection{Skalierung}
Zwischen -1 und 1 oder zwischen 0 und 1

\section{Training}
Übersicht über Training, Weitere Möglichkeiten (K-Means, Kohonenkarten,...)
\subsection{Neural Gas}
Robustes Neuronales Netz
Nachbarschaftsreichweite
Update-Regel
\subsection{Support Vector Machine}
Support Vector Machine (SVM) ist ein (optimaler) Klassifikator
Eingesetzt wird Multi-Class SVM (One vs. One, One vs. Rest)

\section{Prediction}
Übersicht über Prediction, Erklärung Prediction = Arbeitsphase, erzeugt Ergebnis
\subsection{Nearest Neighbor}
Distanz zwischen Eingabevektor und Codebookvektoren
\subsection{Klassifikation}
Einordnung von Merkmalsvektoren

\section{Analysis}
Dient der Überprüfung des Ergebnisses der Prediciton-Phase evtl. mit Visualisierung\\
Möglichkeiten sind die absolute Erkennungsrate, Erkennungsrate auf mehrere Frames hintereinander, Verwechslungsmatrix
