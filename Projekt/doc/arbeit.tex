\NeedsTeXFormat{LaTeX2e}
\documentclass[a4paper,10pt,bibliography=totoc,oneside,openright,numbers=noenddot,headings=normal,DIV=9
%,draft
]{scrreprt}
\KOMAoptions{DIV=last}
\usepackage{scrhack}

\pagestyle{headings}
\usepackage[ngerman]{babel}
\usepackage[babel,german=quotes]{csquotes}
\usepackage[utf8]{inputenc}
\usepackage[T1]{fontenc}
\renewcommand{\sfdefault}{phv}
\renewcommand{\rmdefault}{phv}
\renewcommand{\ttdefault}{pcr}
\usepackage{graphicx}
\usepackage{verbatim}
\usepackage{tabularx}
\usepackage{subfigure}
\usepackage{url}
\usepackage{color}
\definecolor{LinkColor}{rgb}{0,0,0.2}
\usepackage{amssymb}
\usepackage{amsmath}
\usepackage{amsthm}
\usepackage{setspace}
\usepackage{fullpage} % or addmargin in cover
\usepackage{listings}
\lstset{language=Java,
  showstringspaces=false,
  frame=single,
  numbers=left,
  basicstyle=\ttfamily,
  numberstyle=\tiny}

% hier Namen etc. einsetzen
\newcommand{\fullname}{Tobias Dreher}
\newcommand{\email}{tobias.dreher@uni-ulm.de}
\newcommand{\titel}{Sprechererkennung}
\newcommand{\jahr}{2012}
\newcommand{\matnr}{123456}
\newcommand{\gutachterA}{Dr.\ Friedhelm Schwenker}
\newcommand{\betreuer}{Dipl.-Inf.\ Sascha Meudt}
\newcommand{\fakultaet}{Ingenieurwissenschaften\\und Informatik}
\newcommand{\institut}{Institut für Neuroinformatik}

%color in tables
\usepackage{colortbl}
\definecolor{Gray}{rgb}{0.80784, 0.86667, 0.90196} %dunkelblau
\definecolor{Lightgray}{rgb}{0.9176, 0.95, 0.95686} %hellblau
\definecolor{Akzent}{rgb}{0.6627, 0.63529, 0.55294} %akzentfarbe
\setlength{\arrayrulewidth}{0.1pt}

\clubpenalty10000
\widowpenalty10000

\setlength{\parindent}{0pt}
\setlength{\parskip}{1.4ex plus 0.35ex minus 0.3ex}

% Tiefe, bis zu der Überschriften in das Inhaltsverzeichnis kommen
\setcounter{tocdepth}{1}

\pdfinfo{
  /Author (\fullname)
  /Title (\titel)
  /Producer (PDFLaTeX)
  /Keywords ()
}

\usepackage{hyperref}
\hypersetup{
pdftitle=\titel,
pdfauthor=\fullname,
pdfsubject={Projekt},
pdfproducer={PDFLaTeX},
colorlinks=true,
	linkcolor=LinkColor,
	citecolor=LinkColor,
	filecolor=LinkColor,
	menucolor=LinkColor,
	urlcolor=LinkColor,
pdfborder=0 0 0	% keine Box um die Links!
}

%Trennungsregeln
\hyphenation{Sil-ben-trenn-ung}

\begin{document}
%\frontmatter

% Titelseite
\thispagestyle{empty}
%\begin{addmargin*}[-8mm]{8mm}
\vspace*{1.0em}

\includegraphics[height=1.8cm]{images/unilogo_bild}
\hfill
\includegraphics[height=1.8cm]{images/unilogo_wort}
\vspace*{2.1em}

{\footnotesize
{\bfseries Universität Ulm} \textbar ~89069 Ulm \textbar ~Germany
\hfill\parbox[t]{42mm}{\bfseries Fakultät für\\
\fakultaet\\
\mdseries \institut}
\vspace*{2cm}

\parbox{140mm}{\bfseries \huge \titel}

{\footnotesize Projekt an der Universität Ulm}
\vspace*{4em}

{\footnotesize
{\bfseries Vorgelegt von:}\\
\fullname\\\email\\[2em]
{\bfseries Gutachter:}\\
\gutachterA\\
[2em] %\gutachterB\\
{\bfseries Betreuer:}\\
\betreuer
\\[1.5em]
\jahr}
}
%\end{addmargin*}


% ab hier Zeilenabstand 1,4 fach 10pt/14pt
\setstretch{1.4}

\tableofcontents

%\mainmatter
\chapter{Einleitung}
Übersicht über dieses Kapitel

\section{Motivation}
Forensik
Webkonferenzen
Protokollierung

\section{Ziele}
Erkennung in Offline- und Online-Modus
Vergleich verschiedener Algorithmen

\section{Übersicht}
Übersicht über die folgenden Kapitel

\chapter{Theorie}

\section{Übersicht}

\section{Preprocessing}

\section{Training}

\section{Prediction}

\section{Analysis}

\chapter{Implementierung}
\label{cha:implementierung}
Das Ziel der Projektarbeit ist sowohl einen Offline- als auch einen Online-Modus zu implementieren und insbesondere mit dem Offline-Modus die eingesetzten Algorithmen zu vergleichen. Im diesem Kapitel werden die Kernmodule, die für beide Modi benötigt werden, sowie die einzelnen Modi beschrieben. Des Weiteren wird im letzten Abschnitt ein Evolutionärer Algorithmus zur Parameteroptimierung beschrieben.

\section{Kernmodule}
Der Kern sämtlicher Implementierungen stellen verschiedene Module dar, die für verschiedene Einsatzzwecke benutzt werden können. Bei jedem Modul handelt es sich um ein ausführbares Programm. In Abbildung \ref{fig:moduluebersicht} sind alle Kernmodule dargestellt. Gelb markiert sind die eigens für das Projekt in Java implementierten Module, blau markiert sind Module, die aus der \emph{libsvm} (Version 3.12) stammen. Die Kernmodule decken die in Kapitel \ref{cha:theorie} beschrieben vier Arbeitsschritte vollständig ab. 

\begin{description}
	\item[Preprocessing] Im Modul \emph{FeatureExtraction} werden die ankommenden Wav-Dateien (Mono, 16 kHz, 16 Bit) in Featurevektoren umgewandelt. Die Umwandlung erfolgt über LPC oder MFCC, wobei für MFCC die Bibliothek \emph{CoMIRVA} verwendet wird. \cite{bib:comirva} Für die Verwendung von SVM in der Trainings- und Prediction-Phase ist eine Skalierung von Vorteil. Hierzu wird das Modul \emph{SvmScale} der \emph{libsvm} verwendet.
	\item[Training] Das Training kann auf zweierlei Arten durchgeführt werden. Zum einen wird durch das Modul \emph{SvmTrain} der \emph{libsvm} ein SVM-Klassifikator zur Verfügung gestellt. Zum anderen wird durch \emph{CreateCodebook} ein Codebuch auf Basis des Neural Gas Algorithmus erzeugt.
	\item[Prediction] Wurde beim Training SVM verwendet, wird in dieser Phase das Modul \emph{SvmPrediction} der \emph{libsvm} verwendet. Wurde dagegen ein Codebuch erstellt, wird die Methode des nächsten Nachbarn mit dem Modul \emph{NearestNeighbor} verwendet.
	\item[Analysis] Das Modul \emph{Analysis} wertet die Ergebnisse der Prediction-Phase aus und speichert sowohl die Erkennungsrate, als auch eine Verwechslungsmatrix in einer Ausgabedatei.
\end{description}

Zum Informationsaustausch zwischen den Modulen wird das bestehende Dateiformat der \emph{libsvm} benutzt. Dadurch ist die Interoperabilität mit der \emph{libsvm} gewährleistet, ohne dass ein Wrapper dafür benötigt wird.

\begin{figure}[h]
  \centering
  \includegraphics[width=0.75\textwidth]{images/moduluebersicht}
  \caption{Kernmodule in der Übersicht}
  \label{fig:moduluebersicht}
\end{figure}

Der Vorteil in der Modularisierung liegt in der flexiblen Anwendung. So kann beispielsweise unkompliziert getestet werden, ob die Skalierung auch beim Einsatz von \emph{Neural Gas} sinnvoll ist. Des Weiteren können die Module für weitere Einsatzzwecke genutzt werden. Um zum Beispiel eine Gesichts- oder Texterkennung zu implementieren muss nur das \emph{FeatureExtraction}-Modul ausgetauscht werden. Der Nachteil einer solchen Modularisierung stellen die beschränkten Debuggingmöglichkeiten dar. Da es sich um für sich jeweils abgeschlossene Programme handelt, kann man nicht auf herkömmliche Weise im Programmcode debuggen.

\section{Offline-Modus}
Ein Bash-Skript dient der Vergleichbarkeit der Ergebnisse. Dieses ermöglicht die Veränderung sämtlicher Einstellungen, wie die Fensterbreite der Featurevektoren oder der Trainingsmethode. Das Skript führt eine Kreuzvalidierung über drei Blöcke durch und speichert das Analyseergebnis in einer Datei.

\section{Online-Modus}
Ein Java-Programm stellt den Online-Modus der Sprechererkennung dar. Dieses verwendet MFCC in der Preprocessing-Phase und Neural Gas in der Trainingsphase. Diese Algorithmen werden verwendet, da sie zum Zeitpunkt der Implementierung subjektiv am effizientesten arbeiteten. Das Programm ist so implementiert, dass es immer nach 320 ms ein Statusupdate auf der Konsole ausgibt. In diesem wird der Sprecher über die letzten 1,6 Sekunden ermittelt, wobei neuere Werte stärker gewichtet werden. Der Gewinner des neuesten Feature-Vektors wird 5-fach so stark gewichtet, wie der des ältesten Feature-Vektors. Die Gewichtung nimmt linear mit dem Alter der Feature-Vektoren ab.

\section{Evolutionärer Algorithmus zur Parameteroptimierung}
Betrachtet man die einstellbaren Parameter für MFCC und Neural, ergibt dies einen 7-dimensionalen Vektorraum in dem das Optimum der Parameter gefunden werden soll. Die einzelnen Parameter sind:
\begin{itemize}
	\item Fensterbreite
	\item Überlappung der Fenster
	\item Merkmale pro Fenster
	\item Minmale Energielevel
	\item Fensterfunktion (Hamming oder Hann)
	\item Größe des Codebuchs
	\item Iterationen des Neural Gas Algorithmus
\end{itemize}
Sollte die Fensterbreite veränderbar sein, lassen sich die Ergebnisse nur noch schwer vergleichen, da mit der Fensterbreite auch die Information pro Fenster steigt. Somit wurde diese Einstellung auf 32 ms Fensterbreite fixiert.

Um das Optimum in dem noch verbleibenden 6-dimensionalen Vektorraum zu finden, dient ein Evolutionärer Algorithmus. Dabei wird eine variable Mutationsrate eingesetzt, da diese lokale Optimas besser vermeidet, als eine statistische Mutationsrate. Die Fitnessfunktion ist in der Funktion \ref{equ:fitness} dargestellt. Als Eingabeparameter dienen der Funktion die für den Durchlauf mit den aktuellen Parametern benötigte Zeit $t$ in Sekunden, sowie die absolute Erkennungsrate $acc$ in Prozent. Zur Selektion dient eine Tournament-Selektion, bei de je vier Individuen gegen einander antreten, wobei nur der beste in die nächste Population gelangt. Eine Population besteht aus 12 Individuen. 

\begin{equation}
	\label{equ:fitness}
	\begin{aligned}[t]f(t,acc) = \frac{acc}{1 - e^{\left(\frac{-t}{1000}\right)}}\end{aligned}
\end{equation}

\title{Data Mining - Aufgabenblatt 3}
\author{Andra Herta, Tobias Dreher, Dimitrij Zharkov}
\date{19.12.2011}

\documentclass[%
	pdftex,%              PDFTex verwenden da wir ausschliesslich ein PDF erzeugen.
	oneside,%             Einseitiger Druck.
	12pt,%                Grosse Schrift, besser geeignet für A4.
	parskip=half,%        Halbe Zeile Abstand zwischen Absätzen.
	headsepline,%         Linie nach Kopfzeile.
	footsepline,%         Linie vor Fusszeile.
	bibtotocnumbered,%    Literaturverzeichnis im Inhaltsverzeichnis nummeriert einfügen
]{scrartcl}

\usepackage[utf8]{inputenc}
\usepackage[T1]{fontenc}
\usepackage[ngerman]{babel}

\usepackage{amsmath}
\usepackage{amssymb}

\usepackage{libertine}

\usepackage{graphicx}

\usepackage{placeins}
 

\begin{document}
\maketitle

\section*{Aufgabe 9}
\subsection*{Versuch 1}
Datensatz aus Aufgabe 7, Toleranz $\epsilon = 1*e^{-5}$, Fuzzifier $b$ variert zwischen 1.1 und 3:
\begin{figure}[h]
  \begin{minipage}[b]{0.5\linewidth}
    \centering
    \includegraphics[width=1\linewidth]{../img/data7b11.png}
    \caption{$b = 1.1$}
  \end{minipage}
  \begin{minipage}[b]{0.5\linewidth}
    \centering
    \includegraphics[width=1\linewidth]{../img/data7b2.png}
    \caption{$b = 2$}	
  \end{minipage}
\end{figure}
\begin{figure}[h]
    \centering
    \includegraphics[width=0.5\linewidth]{../img/data7b3.png}
    \caption{$b = 3$}
\end{figure}

\FloatBarrier
\subsection*{Versuch 2}
Datensatz aus Aufgabe 7, Fuzzifier $b = 2$, Toleranz $\epsilon$ variert zwischen $1*e^{-2}$ und $1*e^{-10}$:
\begin{figure}[h]
  \begin{minipage}[b]{0.5\linewidth}
    \centering
    \includegraphics[width=1\linewidth]{../img/data7b2e2.png}
    \caption{$\epsilon = 1*e^{-2}$}
  \end{minipage}
  \begin{minipage}[b]{0.5\linewidth}
    \centering
    \includegraphics[width=1\linewidth]{../img/data7b2e5.png}
    \caption{$\epsilon = 1*e^{-5}$}	
  \end{minipage}
\end{figure}
\begin{figure}[h]
    \centering
    \includegraphics[width=0.5\linewidth]{../img/data7b2e10.png}
    \caption{$\epsilon = 1*e^{-10}$}
\end{figure}

\FloatBarrier
\subsection*{Versuch 3}
Datensatz aus Aufgabe 8, Toleranz $\epsilon = 1*e^{-5}$, Clusteranzahl $k = 2$, Fuzzifier $b$ variert zwischen 1.1 und 3:
\begin{figure}[h]
  \begin{minipage}[b]{0.5\linewidth}
    \centering
    \includegraphics[width=1\linewidth]{../img/data8b11e5c2.png}
    \caption{$b = 1.1$}
  \end{minipage}
  \begin{minipage}[b]{0.5\linewidth}
    \centering
    \includegraphics[width=1\linewidth]{../img/data8b2e5c2.png}
    \caption{$b = 2$}	
  \end{minipage}
\end{figure}
\begin{figure}[h]
    \centering
    \includegraphics[width=0.5\linewidth]{../img/data8b3e5c2.png}
    \caption{$b = 3$}
\end{figure}

\FloatBarrier
\subsection*{Versuch 4}
Datensatz aus Aufgabe 8, Toleranz $\epsilon = 1*e^{-5}$, Fuzzifier $b = 2$, Clusteranzahl $k$ variert zwischen 2 und 5:
\begin{figure}[h]
  \begin{minipage}[b]{0.5\linewidth}
    \centering
    \includegraphics[width=1\linewidth]{../img/data8b2e5c2.png}
    \caption{$k = 2$}
  \end{minipage}
  \begin{minipage}[b]{0.5\linewidth}
    \centering
    \includegraphics[width=1\linewidth]{../img/data8b2e5c3.png}
    \caption{$k = 3$}	
  \end{minipage}
\end{figure}
\begin{figure}[h]
    \centering
    \includegraphics[width=0.5\linewidth]{../img/data8b2e5c5.png}
    \caption{$k = 5$}
\end{figure}
\end{document}

\chapter{Diskussion}
\label{cha:diskussion}
Im abschließenden Kapitel werden die Ergebnisse der Projektarbeit analysiert und diskutiert. Hierbei werden beide Modis in je einem Abschnitt berücksichtigt.

\section{Offline-Modus}
Im Offline-Modus werden gute Erkennungsraten mit bis zu 80 \% auf Einzelframes erzielt. Dies führt dazu, dass schon nach wenigen 32ms-Fenstern ein Sprecher sicher identifiziert werden kann. Allerdings kann hier kein fester Wert genannt werden, da es vom Sprecher und den Sprechpausen abhängt. Sollte die Erkennung beispielsweise in einer Sprechpause eingesetzt werden, würde diese durch das hohe Energielevel ausgeschnitten und es fände keine Identifizierung eines Sprechers statt.

Bei den Preprocessing-Algorithmen lässt MFCC LPC deutlich hinter sich. So erzielen die Kombinationen mit MFCC zwischen 26 und 31 Prozentpunkte mehr als die Kombinationen mit LPC.

Bei den Trainings-Algorithmen ist SVM mit einem linearen Kernel geringfügig besser als Neural Gas. Wobei dies beim Einsatz von LPC noch einen Unterschied von 6 Prozentpunkten ausmacht, ist es bei MFCC nur noch ein Prozentpunkt. Nach der Optimierung durch den Evolutionären Algorithmus hat sich dieser Vorsprung allerdings ins negative verkehrt. Der Einsatz vom RBF-Kernel bei SVM verzeichnet nochmal einen deutlich Sprung gegenüber den zwei anderen Trainings-Methoden. So ist SVM mit dem RBF-Kernel die Trainingsmethode mit der höchsten Erkennungsrate auf Einzelframes. In Kombination mit MFCC erzielt SVM (RBF-Kernel) somit die beste Erkennungsrate, die in dieser Projektarbeit erreicht werden konnte.

Abgesehen von der Erkennungsrate spielen in der Praxis auch weitere Faktoren eine Rolle. Sollte Skalierbarkeit wichtig sein, zum Beispiel beim Einsatz auf einer Internetplattform, so scheint die Kombination MFCC und Neural Gas am geeignetesten zu sein. Denn beim Training von SVM wird mit allen Daten trainiert. Dies führt dazu, dass bei einer stetig wachsenden Sprecheranzahl ständig neu und mit größeren Datensätzen trainiert werden muss. Bei Neural Gas dagegen kann für einen neu hinzugekommenen Sprecher einfach ein neues Codebuch erzeugt werden und zu den restlichen Codebüchern hinzugefügt werden. Allerdings wurde in dieser Projektarbeit nicht überprüft wie sich das System mit einer großen Anzahl an Sprechern verhält.

\section{Online-Modus}
Im Online-Modus konnten nur schlechte bis mittelmäßige Erkennungsraten erzeugt werden. Diese sind zu unzuverlässig, um damit arbeiten zu können. Folgende Probleme bestehen bei der Online-Erkennung:
\begin{description}
	\item[Umwelt- und Mikrofonunterschiede] Jedes Mikrofon hat seine eigene Charistika und erzeugt somit unterschiedliche Aufnahmen. Ebenso sind Hintergrundgeräusche und akustische Störungen wie Hall nicht zu vermeiden. Somit entsteht eine Diskrepanz zwischen den Trainings- und Prediction-Phase, die zu einer schlechten Erkennungrate führen kann.
	\item[Geringe Trainingsmenge] Für den Vergleich mit einem Laptopmikrofon wurden nur geringe Trainingsmengen aufgenommen. Diese sind wohl zu gering, um die Sprecher korrekt voneinander zu unterscheiden.
	\item[Fehlende Skalierung] Die Eingangsdaten wurden nicht skaliert. Da unterschiedliche Mikrophone und unterschiedliche Sitzpositionen vor dem Mikrofon zu unterschiedlichen Lautstärken führen, wäre dies allerdings notwendig gewesen um stabilere Ergebnisse zu erzielen.
	\item[Fehlendes Energielevel] Ebenso scheint das fehlende Energielevel ein gewichtiger Grund, dass keine guten Erkennungsraten erzielt wurden. So wurde in den Offline-Tests nachgewiesen, dass ein hohes Energielevel auch hohe Erkennungsraten zur Folge haben.
\end{description}


\appendix
% hier Anhänge einbinden
%\input{chapters/sources}

%\backmatter

%\bibliographystyle{plaindin} % Nummern und alphabetisch sortiert
%\bibliographystyle{alphadin} % Buchstaben und sortiert
%\bibliographystyle{abbrvdin} % Nummern und abgekürzte Namen
\bibliographystyle{unsrtdin} % Nummern und unsortiert
\bibliography{bibliography}


%\clearpage
%\thispagestyle{empty}

%Name: \fullname \hfill Matrikelnummer: \matnr \vspace{2cm}

%\minisec{Erklärung}

%Ich erkläre, dass ich die Arbeit selbständig verfasst und keine anderen als die angegebenen Quellen und Hilfsmittel verwendet habe.\vspace{2cm}

%Ulm, den \dotfill

%\hspace{10cm} {\footnotesize \fullname}
\end{document}
